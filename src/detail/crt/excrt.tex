\begin{frame}{扩展中国剩余定理 (exCRT)}
	\label{cgrunt:ssec:excrt}

	\only<1-3>{为什么 CRT 要求模数必须两两互素?

		因为我们在 CRT 的证明过程中即可发现, 其关键便是构造这样的方程组

		\[
			\begin{cases}
				x\equiv1\pmod{m_i} \\
				x\equiv0\pmod{m_j}~(j\ne i)
			\end{cases}
		\]}

	\only<2-3>{此时的 \(x=M_iN_i\), \(M_i=\prod_{j=1,j\ne i}^km_j\), 而 \(N_i\) 是 \(M_i\) 关于 \(m_i\) 的 \textbf{逆元}}

	\only<3>{我们知道, 如果一个整数 \(a\) 在模 \(m\) 意义下有逆元, 其前提之一便是 \((a,m)=1\), 在此处便是 \((M_i,m_i)=1\)

		故由最大公约数的性质, 此处的 \(k\) 个模数 \(m_1,m_2,\dots,m_k\) 必须两两互素}

	\only<4>{那么我们如何摆脱这个限制条件呢? 这就要求我们换一种构造方式}

	\only<5>{我们观察下面的式子

		\[
			\begin{cases}
				x\equiv b_1\pmod{m_1} \\
				x\equiv b_2\pmod{m_2}
			\end{cases}
		\]}

	\only<6->{由同余定义 (\ref{cgrunt:def:cgrunt}), 我们有

		\[
			\begin{cases}
				m_1\mid x-b_1 \\
				m_2\mid x-b_2
			\end{cases}
		\]

		即存在整数 \(k_1,k_2\) 使得 \(x=m_1k_1+b_1=m_2k_2+b_2\)}

	\only<7->{整理一下, 便有
		\begin{equation}
			m_1k_1-m_2k_2=b_2-b_1
		\end{equation}

		此时我们便可以通过 exgcd 解决了}
	\only<8->{, 求出\(k_1,k_2\)之后便得到

		\begin{equation}
			x\equiv m_1k_1+b_1\equiv m_2k_2+b_2\pmod{[m_1,m_2]}
		\end{equation}}
\end{frame}
